\documentclass[a4paper,12pt]{article}
\usepackage[utf8]{inputenc}
\usepackage[T1]{fontenc}
\usepackage[ngerman]{babel}
\usepackage{geometry}
\usepackage{fancyhdr}
\usepackage{enumitem}
\usepackage{titlesec}

% Page Layout Configuration
\geometry{top=3cm, bottom=3cm, left=2.5cm, right=2.5cm}

% Header and Footer Setup
\pagestyle{fancy}
\fancyhf{}
\lhead{\textbf{Prog 2} \\ Objektorientierte Programmierung in Java}
\rhead{\textbf{ZUSATZAUFGABEN} \\ (Zur Vorbereitung)}
\cfoot{- \thepage \ -}
\renewcommand{\headrulewidth}{0.4pt}

% Title Formatting
\titleformat{\section}{\large\bfseries}{}{0em}{}

\begin{document}

\section*{Übungsaufgabe 1: Gruppenprojekt „Das Aquarium“}
\textbf{Thema:} GUI, Graphics, Vektoren \& Multithreading \\

[cite_start]Diese Aufgabe baut auf dem „simplen Zeichenprogramm“ [cite: 110] auf, fügt aber eine dynamische Komponente hinzu.

\subsection*{A: Das Aquarium}
Erstellen Sie eine Anwendung, die ein Aquarium simuliert.
\begin{itemize}
    \item Es gibt einen \texttt{JFrame} mit einem \texttt{JPanel} (die Wasserfläche).
    \item Durch einen Mausklick (Left-Click) an eine beliebige Stelle soll ein neuer „Fisch“ erzeugt werden.
    \item Ein „Fisch“ ist grafisch einfach eine Ellipse oder ein Kreis (Farbe zufällig).
\end{itemize}

\subsection*{B: Die Logik (Threads)}
\begin{itemize}
    \item Jeder Fisch soll nicht nur als Daten-Objekt im Vektor existieren, sondern auch ein \textbf{eigener Thread} sein (oder von einem Timer gesteuert werden).
    \item Die Fische sollen sich horizontal von links nach rechts bewegen.
    \item Trifft ein Fisch auf den rechten Rand, soll er am linken Rand wieder erscheinen.
    [cite_start]\item \textbf{Wichtig:} Das Panel muss regelmäßig \texttt{repaint()} aufrufen, damit die Bewegungen sichtbar werden[cite: 106].
\end{itemize}

\subsection*{Tipps zur Struktur}
\begin{itemize}
    \item Klasse \texttt{Fisch}: Speichert x, y, Farbe und Geschwindigkeit. Implementiert das Interface \texttt{Runnable}.
    [cite_start]\item Klasse \texttt{AquariumPanel}: Besitzt den \texttt{Vector<Fisch>}[cite: 128]. Die \texttt{paintComponent} Methode iteriert über den Vektor und zeichnet alle Fische an ihren \textit{aktuellen} Positionen.
\end{itemize}

\vspace{1cm}
\hrule
\vspace{1cm}

\section*{Übungsaufgabe 2: „Fenster-Panik“ (FrameChaos)}
\textbf{Thema:} JFrame, Threading \& Randomization \\

[cite_start]Diese Aufgabe ist das Gegenstück zum „FrameMover“[cite: 144], erfordert aber Interaktion. \\

Implementieren Sie eine Klasse \texttt{PanicFrame}.
\begin{itemize}
    \item Das Programm startet \textbf{einen} einzigen Frame (Größe 300x300).
    \item In diesem Frame befindet sich ein Button mit der Aufschrift „Klick mich!“.
    \item Der Button hat einen \texttt{MouseListener}.
\end{itemize}

\textbf{Die Logik:}
\begin{itemize}
    \item Versucht der Nutzer, mit der Maus über den Button zu fahren (\texttt{mouseEntered}) oder ihn zu klicken, soll der \textbf{gesamte Frame} sofort an eine zufällige andere Position auf dem Bildschirm springen (\texttt{setLocation}).
    [cite_start]\item Gleichzeitig soll sich bei jedem Sprung die Hintergrundfarbe des Panels zufällig ändern[cite: 146].
\end{itemize}

\textbf{Zusatzaufgabe (Threads):} \\
Starten Sie im Hintergrund einen Thread, der die Größe des Fensters alle 500ms zufällig zwischen 100x100 und 400x400 verändert, um den Nutzer zusätzlich zu verwirren.

\newpage

\section*{Übungsaufgabe 3: Rekursion \& Fraktale}
[cite_start]\textbf{Thema:} Rekursion und Graphics (analog zu den Aufgaben im Skript [cite: 153])

\subsection*{A: Rekursives Zeichnen (Der „Möchtegern-Baum“)}
Schreiben Sie eine Klasse \texttt{FractalPanel} (erbt von \texttt{JPanel}). \\
Implementieren Sie eine rekursive Methode: \\
\texttt{void drawBranch(Graphics g, int x, int y, int length, int angle)}

\begin{itemize}
    \item Die Methode zeichnet eine Linie (den Ast) der Länge \texttt{length} vom Punkt $(x,y)$ in Richtung \texttt{angle}.
    \item Am Ende des Astes ruft sich die Methode \textbf{zweimal} selbst auf:
    \begin{enumerate}
        \item Einmal mit etwas kürzerer Länge und einem Winkel nach links.
        \item Einmal mit etwas kürzerer Länge und einem Winkel nach rechts.
    \end{enumerate}
    \item \textbf{Abbruchbedingung:} Wenn die Länge kleiner als 2 Pixel ist, stoppt die Rekursion.
    \item Starten Sie die Zeichnung in der \texttt{paintComponent} Methode am unteren Bildschirmrand.
\end{itemize}

\subsection*{B: Mathematische Rekursion}
Implementieren Sie eine statische Methode: \\
\texttt{boolean isPalindrome(String s)}

\begin{itemize}
    \item Die Methode soll \textbf{rekursiv} prüfen, ob ein String vorwärts wie rückwärts gleich ist (z.B. „Lagerregal“ oder „Rentner“).
    \item \textit{Tipp:} Vergleichen Sie das erste und letzte Zeichen. Wenn diese gleich sind, schneiden Sie sie ab und rufen die Methode mit dem Rest-String erneut auf.
\end{itemize}

\end{document}
