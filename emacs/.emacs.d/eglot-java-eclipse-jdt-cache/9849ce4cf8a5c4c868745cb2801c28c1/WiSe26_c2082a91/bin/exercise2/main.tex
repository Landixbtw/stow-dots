\documentclass[a4paper,11pt]{article}

% --- Pakete für Sprache und Kodierung ---
\usepackage[utf8]{inputenc}
\usepackage[T1]{fontenc}
\usepackage[ngerman]{babel}

% --- Layout und Ränder ---
\usepackage{geometry}
\geometry{top=3cm, bottom=3cm, left=2.5cm, right=2.5cm}
\usepackage{fancyhdr}
\usepackage{parskip} % Besserer Abstand zwischen Absätzen

% --- Mathe und Code ---
\usepackage{amsmath}
\usepackage{listings}
\usepackage{xcolor}

% --- Kopf- und Fußzeilen Konfiguration ---
\pagestyle{fancy}
\fancyhf{} % Alles bereinigen
\lhead{\textbf{Prog 2}\\Objektorientierte Programmierung in Java}
\rhead{\textbf{ZUSATZAUFGABEN}\\(Prüfungsvorbereitung)}
\cfoot{\thepage}
\renewcommand{\headrulewidth}{0.4pt}

% --- Code-Stil Definition ---
\lstset{
    basicstyle=\ttfamily\small,
    frame=single,
    breaklines=true,
    backgroundcolor=\color{gray!10},
    literate={ö}{{\"o}}1 {ä}{{\"a}}1 {ü}{{\"u}}1 {ß}{{\ss}}1
}

\begin{document}

% --- Titelbereich ---
\begin{center}
    \LARGE \textbf{Übungsblatt: Prüfungsvorbereitung}
\end{center}
\vspace{0.5cm}

% ================= AUFGABE 1 =================
\section*{Übungsaufgabe 1: Klassen, Collections \& File I/O}
\textit{Themen: Szenario Restaurant, ArrayList/Vector, Exception Handling}

\subsection*{A: Die Bestellung}
Schreiben Sie eine Klasse \texttt{Bestellung}. Jede Bestellung verfügt über folgende Attribute:
\begin{itemize}
    \item Einen \texttt{Namen} (des Gerichts, Typ String)
    \item Einen \texttt{Preis} (Typ double)
    \item Eine \texttt{TischNummer} (Typ int)
\end{itemize}

\subsection*{B: Die Verwaltung}
Schreiben Sie eine Klasse \texttt{Kueche}.
\begin{itemize}
    \item Diese nutzt intern einen \textbf{Vektor} (oder eine \texttt{ArrayList}), um Objekte vom Typ \texttt{Bestellung} zu speichern.
    \item Implementieren Sie eine Methode: \\
    \texttt{public void neueBestellung(String name, double preis, int tisch)} \\
    Diese Methode soll ein neues Objekt erzeugen und in die Liste einfügen.
\end{itemize}

\subsection*{C: Dateiverarbeitung (File I/O)}
Implementieren Sie in der Klasse \texttt{Kueche} eine Methode: \\
\texttt{public void tagesAbschlussSpeichern(String dateiname)}
\begin{itemize}
    \item Die Methode soll alle aktuellen Bestellungen in eine Textdatei schreiben.
    \item Formatbeispiel in der Datei: \texttt{"Tisch 1: Pizza - 9.50 Euro"}
    \item Fangen Sie mögliche Fehler (Exceptions) mit einem Standard \texttt{try-catch}-Block ab (keine eigene Exception-Klasse schreiben!).
\end{itemize}

\vspace{0.5cm}
\hrule
\vspace{0.5cm}

% ================= AUFGABE 2 =================
\section*{Übungsaufgabe 2: Parallele Programmierung}
\textit{Themen: Threads, Runnable, sleep}

Schreiben Sie eine Klasse \texttt{Koch}, die von \texttt{Thread} erbt (oder \texttt{Runnable} implementiert).
\begin{enumerate}
    \item Ein Koch wird mit einem \texttt{GerichtNamen} (String) und einer \texttt{ZubereitungsZeit} (int, in Millisekunden) initialisiert.
    \item In der \texttt{run()}-Methode soll der Koch folgende Schritte ausführen:
    \begin{itemize}
        \item Konsolenausgabe: \texttt{"Fange an mit [Gericht]..."}
        \item Schlafen für die angegebene Zeit (\texttt{Thread.sleep()}).
        \item Konsolenausgabe: \texttt{"[Gericht] ist fertig!"}
    \end{itemize}
    \item \textbf{Main-Methode:} Erstellen Sie in einer Hauptklasse drei Koch-Objekte mit unterschiedlichen Zeiten und starten Sie diese so, dass sie \textbf{parallel} arbeiten.
\end{enumerate}

\newpage

% ================= AUFGABE 3 =================
\section*{Übungsaufgabe 3: Rekursion}
\textit{Themen: Rekursive Aufrufe, Logik, Konsolenausgabe}

\subsection*{A: Die Quersumme}
Schreiben Sie eine \textbf{rekursive} Funktion \texttt{int quersumme(int n)}. \\
Beispiel: \texttt{quersumme(123)} $\rightarrow 1 + 2 + 3 = 6$. \\
\textit{Tipp: Nutzen Sie Modulo 10 und Ganzzahldivision durch 10.}

\subsection*{B: Rekursives Muster}
Schreiben Sie eine rekursive Methode \texttt{printPattern(int n)}, die folgende Ausgabe auf der Konsole erzeugt (Beispiel für $n=4$):

\begin{lstlisting}
*
**
***
****
***
**
*
\end{lstlisting}
\textit{Hinweis: Die Methode ruft sich selbst in der Mitte auf. Reihenfolge: Print, rekursiver Aufruf, Print.}

\vspace{0.5cm}
\hrule
\vspace{0.5cm}

% ================= AUFGABE 4 =================
\section*{Übungsaufgabe 4: Grafik \& Logik}
\textit{Themen: JPanel, paintComponent, Mathe-Logik}

Erstellen Sie eine Klasse \texttt{SternenPanel}, die von \texttt{JPanel} erbt.

\subsection*{A: Die Datenhaltung}
\begin{itemize}
    \item Das Panel besitzt eine Liste (Vektor oder ArrayList) von \texttt{Point}-Objekten.
    \item Erzeugen Sie im Konstruktor 50 zufällige Punkte $(x, y)$ innerhalb der Panel-Größe und speichern Sie diese in der Liste.
\end{itemize}

\subsection*{B: Das Zeichnen (paintComponent)}
Überschreiben Sie die \texttt{paintComponent(Graphics g)} Methode:
\begin{enumerate}
    \item Zeichnen Sie den Hintergrund schwarz.
    \item Zeichnen Sie jeden Punkt aus der Liste als kleinen weißen Kreis.
    \item \textbf{Die Logik:} Überprüfen Sie in einer doppelten Schleife den Abstand aller Punkte zueinander.
    \item Wenn der Abstand zwischen Punkt A und Punkt B kleiner als 50 Pixel ist, zeichnen Sie eine \textbf{gelbe Linie} zwischen ihnen.
\end{enumerate}
\textit{Formel für die Distanz:}
\[
\text{Distanz} = \sqrt{(x_2 - x_1)^2 + (y_2 - y_1)^2}
\]
(Alternativ nutzen Sie \texttt{Point.distance(...)})

\vspace{0.5cm}
\hrule
\vspace{0.5cm}

% ================= AUFGABE 5 =================
\section*{Übungsaufgabe 5: Theorie-Check}
Beantworten Sie folgende Fragen stichpunktartig:

\begin{enumerate}
    \item \textbf{Vektor vs. Array:} Nennen Sie zwei wesentliche Unterschiede zwischen einem primitiven Array (\texttt{[]}) und der Klasse \texttt{Vector}.
    \item \textbf{Packages:} Wozu dienen Packages in Java? Nennen Sie einen Vorteil.
    \item \textbf{Exceptions:} Was ist der Unterschied zwischen einer \textit{Checked Exception} (z.B. \texttt{IOException}) und einer \textit{Unchecked Exception} (z.B. \texttt{NullPointerException}) in Bezug auf den Compiler?
    \item \textbf{Vererbung:} Kann eine Klasse von mehreren abstrakten Klassen erben? Kann sie mehrere Interfaces implementieren?
\end{enumerate}

\end{document}
